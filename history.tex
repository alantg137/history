\documentclass[12pt]{article}

%
%Margin - 1 inch on all sides
%
\usepackage[letterpaper]{geometry}
\usepackage{times}
\geometry{top=1.0in, bottom=1.0in, left=1.0in, right=1.0in}

%
%Doublespacing
%
\usepackage{setspace}
\doublespacing

%
%Fancy-header package to modify header/page numbering (insert last name)
%
\usepackage{fancyhdr}
\pagestyle{fancy}
\lhead{}
\chead{}
\rhead{Grayson \thepage}
\lfoot{}
\cfoot{}
\rfoot{}
\renewcommand{\headrulewidth}{0pt}
\renewcommand{\footrulewidth}{0pt}
%To make sure we actually have header 0.5in away from top edge
%12pt is one-sixth of an inch. Subtract this from 0.5in to get headsep value
\setlength\headsep{0.333in}

%
%Works cited environment
%(to start, use \begin{workscited...}, each entry preceded by \bibent)
% - from Ryan Alcock's MLA style file
%
\newcommand{\bibent}{\noindent \hangindent 40pt}
\newenvironment{workscited}{\newpage \begin{center} Works Cited \end{center}}{\newpage }


%
%Begin document
%
\begin{document}
\begin{flushleft}

%%%%First page name, class, etc
Alan Grayson\\
Daddy\\
Modern History\\
May 7, 2025\\


%%%%Title
\begin{center}
The Unseen Toll of the Irish Potato Famine
\end{center}

%%%%Changes paragraph indentation to 0.5in
\setlength{\parindent}{0.5in}
%%%%Begin body of paper here

The Irish potato famine was catastrophic, but there were some ways the
English could have helped the Irish.  Robert Peel was the Prime
minister during the Blight, and he was trying to stop the starving in
Ireland by repealing the taxes. However, the English did not listen to
him well. They had misunderstood because they did not see the
unseen. If the English had listened to Robert Peel better, the Irish
Potato Famine would not have been so catastrophic. Using Frédéric
Bastiat's reasoning from his essay ``seen and unseen'', we will show
that if the taxes had been repealed more quickly, the famine would not
have been so catastrophic.


From 1845 to 1852 the Irish farmers' potatoes were rotting and the
rest of their crops were being sent to England for rent. There was
also a huge tariff on food from other countries, which meant to make
the Irish buy food from the English farmers. Before the
blight came to Ireland, the Irish farmers had been mostly relying on
their potatoes to feed themselves. When the blight came to Ireland,
now the Irish people's only choice was to buy food from England. As
everyone was racing to get food from the English farmers, this caused
the prices to rise until they were too expensive for the Irish. Then
they didn't have any food to eat.



Robert Peel (1788-1840) had been repeatedly asking the British
parliment to repeal the Corn Laws (the tariff) so the Irish could buy
cheap food, but when he heard the news that the blight had arrived in
Ireland, he tried even harder to convince parliment to agree as a
special case for the potato famine to help the Irish. When they
finally agreed, Robert Peel was unpopular with the English farmers and
was kicked out his position as prime minister. The next prime minister
wasn't as helpful and even remarked that the Irish were starving
because they weren't working hard enough to help themselves.



The reason it took so long for parliment to agree to repeal the taxes
was because it looked like the Corn Laws were good because the English
farmers got more money, but as Bastiat shows in his essay, there are two
unseen parts. First, the Irish lost whatever the English gained.
Second, the Irish could have bought something with that extra money
they spent, but they did not get anything better. Overall, this means
the Corn Laws were bad. However, most of the English only looked at
their benefit instead of the Irish toll.

Eventually, the potatoes stopped rotting. However, by that time, about
a million people had starved, and another million had emigrated. If
the English had listened better, then there would have been tons of
cheap food that the Irish could have bought to survive during the
potato famine. The lesson we should learn is: \textit{always consider the
unseen.}
%%%%Works cited
\begin{workscited}

\bibent
Bauer, Susan Wise. \textit{The Story of the World Vol. 4: The Modern Age.} Well-Trained Mind Press, 2005.

\bibent
Frédéric Bastiat. ``That Which is Seen, and That Which is Not Seen.'' 1850. http://bastiat.org/en/twisatwins.html. Accessed 9 May 2025.



\end{workscited}

\end{flushleft}
\end{document}
\}
