\documentclass[12pt]{article}

%
%Margin - 1 inch on all sides
%
\usepackage[letterpaper]{geometry}
\usepackage{times}
\geometry{top=1.0in, bottom=1.0in, left=1.0in, right=1.0in}

%
%Doublespacing
%
\usepackage{setspace}
\doublespacing

%
%Fancy-header package to modify header/page numbering (insert last name)
%
\usepackage{fancyhdr}
\pagestyle{fancy}
\lhead{}
\chead{}
\rhead{Grayson \thepage}
\lfoot{}
\cfoot{}
\rfoot{}
\renewcommand{\headrulewidth}{0pt}
\renewcommand{\footrulewidth}{0pt}
%To make sure we actually have header 0.5in away from top edge
%12pt is one-sixth of an inch. Subtract this from 0.5in to get headsep value
\setlength\headsep{0.333in}

%
%Works cited environment
%(to start, use \begin{workscited...}, each entry preceded by \bibent)
% - from Ryan Alcock's MLA style file
%
\newcommand{\bibent}{\noindent \hangindent 40pt}
\newenvironment{workscited}{\newpage \begin{center} Works Cited \end{center}}{\newpage }


%
%Begin document
%
\begin{document}
\begin{flushleft}

%%%%First page name, class, etc
Alan Grayson\\
Daddy\\
Modern History\\
May 7, 2025\\


%%%%Title
\begin{center}
The Unseen Toll of the Irish Potato Famine
\end{center}

%%%%Changes paragraph indentation to 0.5in
\setlength{\parindent}{0.5in}
%%%%Begin body of paper here

The Irish potato famine was catastrophic, but there were some ways the
English could have helped the Irish.  Robert Peel was the Prime
minister during the Blight, and he was trying to stop the starving in
Ireland by repealing the taxes. However, the English didn't listen to
him well. They had misunderstood because they didn't see the
unseen. If the English had listened to Robert Peel better, the Irish
Potato Famine wouldn't have been so catastrophic. Using Frédéric
Bastiat's reasoning from his essay ``seen and unseen'', we can show
that if the taxes should have been repealed quicker and the famine
wouldn't have been so deadly.


The Irish farmers' potatoes were rotting and the rest of their crops
were being sent to England for rent. There was also a huge tariff on
food from other countries, which meant to make the Irish buy food from
the English farmers. In 1845, before the blight came to Ireland, the
Irish farmers had been mostly relying on their potatoes to feed
themselves. When the blight came to Ireland, now the Irish people's
only choice was to buy food from England. As everyone was racing to
get food from the English farmers, this caused the prices to rise until
they were too expensive for the Irish. Then they didn't have
any food to eat.



Robert Peel(1788-1840) had been repeatedly asking the British
parliment to repeal the tarrifs so the Irish could buy cheap food, but
when he heard the news that the blight had arrived in Ireland, he
tried even harder to convince Parliment to agree as a special case for
the potato famine so the Irish could get cheap food. When they finally
agreed, Robert Peel was unpopular with the English farmers and was
kicked out his position as Prime Minister. The next Prime minister
wasn't as helpful and even remarked that the Irish were starving
because they weren't working hard enough to help themselves.



The reason it took so long for Parliment to agree was because to the
them it looked like the Corn Laws(the tariff) were good because the
English farmers got more money, but there were $2$ unseen parts,
too. They were that the extra money the English gained cancled with
what the extra money the Irish lost, and the Irish could've bought
something with that extra money they spent, but they didn't get
anything better.  That meant the Corn Laws were bad. However, most of
the English only looked at their benefit instead of the Irish's toll.
Eventually, the potatoes stopped rotting. However, by that time, a
million people starved, and another million starved. If the English
had listened better, then there would've been tons of cheap food that the Irish could've bought to survive during the potato famine.
%%%%Works cited
\begin{workscited}

\bibent
Bauer, Susan Wise. \textit{The Story of the World Vol. 4: The Modern Age.} Well-Trained Mind Press, 2005.

\bibent
Frédéric Bastiat. ``That Which is Seen, and That Which is Not Seen.'' 1850. http://bastiat.org/en/twisatwins.html. Accessed 9 May 2025.



\end{workscited}

\end{flushleft}
\end{document}
\}
