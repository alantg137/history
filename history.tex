\documentclass[12pt]{article}

%
%Margin - 1 inch on all sides
%
\usepackage[letterpaper]{geometry}
\usepackage{times}
\geometry{top=1.0in, bottom=1.0in, left=1.0in, right=1.0in}

%
%Doublespacing
%
\usepackage{setspace}
\doublespacing

%
%Fancy-header package to modify header/page numbering (insert last name)
%
\usepackage{fancyhdr}
\pagestyle{fancy}
\lhead{}
\chead{}
\rhead{Grayson \thepage}
\lfoot{}
\cfoot{}
\rfoot{}
\renewcommand{\headrulewidth}{0pt}
\renewcommand{\footrulewidth}{0pt}
%To make sure we actually have header 0.5in away from top edge
%12pt is one-sixth of an inch. Subtract this from 0.5in to get headsep value
\setlength\headsep{0.333in}

%
%Works cited environment
%(to start, use \begin{workscited...}, each entry preceded by \bibent)
% - from Ryan Alcock's MLA style file
%
\newcommand{\bibent}{\noindent \hangindent 40pt}
\newenvironment{workscited}{\newpage \begin{center} Works Cited \end{center}}{\newpage }


%
%Begin document
%
\begin{document}
\begin{flushleft}

%%%%First page name, class, etc
Alan Grayson\\
Daddy\\
Modern History\\
May 7, 2025\\


%%%%Title
\begin{center}
The Irish Potato Famine
\end{center}

%%%%Changes paragraph indentation to 0.5in
\setlength{\parindent}{0.5in}
%%%%Begin body of paper here

The Irish potato famine was very deadly, but there were some ways the
English could've helped the Irish.  Robert Peel the current Prime minister during the Blight, and he was
trying to stop the starving going on in Ireland. However, the English
didn't litsen to him well. Maybe if the English had listened to Robert Peel
better, the Irish Potato Famine wouldn't have been so deadly.


The Irish farmers potatoes were rotting and the rest of their crops
were being sent to England for rent. There was also a huge tariff on
food from other countries which was supposed to make the Irish buy
food from the English farmers. The Irish farmers had been completly
relying on thier potatoes to feed themselves. When the blight came to
Ireland, now the Irish people's only choice was to buy food from
England. As everyone was racing to get food from the English farmers,
this caused the prices to slowly creep higher and higher till they
were too expensive for the Ireland farmers. Now they didn't have any
food to eat...



Robert Peel(1788-1840) had been repeatedly asking the British
parliment to repeal the tarrifs so the Irish could buy cheap food, but
when he saw the blight, he tried even harder to convince Parliment to
agree.  Once they finally agreed, Robert Peel was now unpopular with
the English farmers and was kicked out his position as Prime
Minister. The next Prime minister wasn't as helpful and even remarked
that the Irish were starving because they weren't working hard enough
to help themselves.



It looked like the Corn Laws(the Tariff) were good because the English farmers got
more money, but there were $2$ unseen parts, too. They were that the
extra money the English gained cancled with what the extra moneyIrish lost, and the
Irish could've bought something with that extra money they spent, but
they didn't get anything better.

%%%%Works cited
\begin{workscited}

\bibent
Allen, R.L. \textit{The American Farm Book; or Compend of American Agriculture; Being a Practical Treatise on Soils, Manures, Draining, Irrigation, Grasses, Grain, Roots, Fruits, Cotton, Tobacc0o, Sugar Cane, Rice, and Every Staple Product of the United States with the Best Methods of Planting, Cultivating, and Prep aration for Market.} New York: Saxton, 1849. Print.

\bibent
Baker, Gladys L., Wayne D. Rasmussen, Vivian Wiser, and Jane M. Porter. \textit{Century of Service: The First 100 Years of the United States Department of Agriculture.}[Federal Government], 1996. Print.

\bibent
Danhof, Clarence H. \textit{Change in Agriculture: The Northern United States, 1820-1870.} Cambridge: Harvard UP, 1969. Print.


\end{workscited}

\end{flushleft}
\end{document}
\}
