\documentclass[12pt]{article}

\begin{document}
\title{Ultrasound and buzzer}
\maketitle
\section{Buzzer}
Hi! this section is about progamming buzzers.
First connect the buzzer to the pico.
Then Import time.
Define the buzzer pin in the program like this:
\begin{verbatim}
buzzer = machine.PWM(machine.Pin((the pin of your buzzer,machine.Pin.OUT))
\end{verbatim}
After that start your loop.
Your program should look something like this:
\begin{verbatim}
import time
buzzer = machine.PWM(machine.Pin(7,machine.Pin.OUT))

while True:
\end{verbatim}
Then set your frequency and loudness like this:
\begin{verbatim}
buzzer.freq(the frequency you want)
buzzer.duty_u16(the volume from 0-65535)
time.sleep_ms(5)
buzzer.duty_u16(0)
\end{verbatim}
Actually 65535/2 is the loudest you can get because you need some time it is off.
The loudness command just says how long it is on.
Remember to indent your lines!
The program should look like this:
\begin{verbatim}
import time
buzzer = machine.PWM(machine.Pin(7,machine.Pin.OUT))

while True:
    buzzer.freq(100)
    buzzer.duty_u16(100)
    time.sleep_ms(5)
    buzzer.duty_u16(0)
\end{verbatim}
And it works!!
\section{Ultrasound}
Connect the pico to the ultrasound.
For this you also import time.
Define your trigger and echo like this:
\begin{verbatim}
trg = machine.Pin(your pin,machine.Pin.OUT)
echo = machine.Pin(your pin,machine.Pin.IN)
\end{verbatim}
Make sure to connect echo and trg to different yellow pins.
Start your loop.
Turn it off for 2 microseconds, on for 10 microseconds, off like this:
\begin{verbatim}
    trg.value(0)
    time.sleep_us(2)
    trg.value(1)
    time.sleep_us(10)
    trg.value(0)
\end{verbatim}
This will get it to shoot out ultrasound which will bounce back.
Then do this:
\begin{verbatim}
    while echo.value() == 0:
        pass
    t1 = time.ticks_us()
    while echo.value() == 1:
        pass
    t2 = time.ticks_us()
\end{verbatim}
This gets the pulse and sets two variables where the 2nd one - the 1st one is the distance
Then you print the difference of the two variables like this:
\begin{verbatim}
print(time.ticks_diff(t2,t1)
\end{verbatim}
This prints the difference of t2 - t1.
Then add a 1 second sleep at the end so that it doesn't print too fast.
The program should be like this:
\begin{verbatim}
import time
trg = machine.Pin(22,machine.Pin.OUT)
echo = machine.Pin(21,machine.Pin.IN)
while True:
    trg.value(0)
    time.sleep_us(2)
    trg.value(1)
    time.sleep_us(10)
    trg.value(0)

    while echo.value() == 0:
        pass
    t1 = time.ticks_us()
    while echo.value() == 1:
        pass
    t2 = time.ticks_us()

    print(time.ticks_diff(t2,t1))
    time.sleep(1)
\end{verbatim}
The program should print out a bunch of numbers which are the distance of the object.
\end{document}
