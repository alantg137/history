\documentclass[12pt]{article}

\usepackage{hyperref}
\usepackage{tabstackengine,xcolor,rotating}

\newcommand\blackcard[2]{%
  \begingroup\fboxsep=0pt\relax
  \fbox{\tabbedCenterstack{%
  \scriptsize$#2$ && \\&\makebox[10pt]{#1}&\\&
  &\rotatebox[origin=c]{180}{\scriptsize$#2$}}}%
  \endgroup}
\newcommand\redcard[2]{%
  \begingroup\fboxsep=0pt\relax
  \fbox{\color{red}\tabbedCenterstack{%
  \scriptsize$#2$ && \\&\makebox[10pt]{#1}&\\&&\rotatebox[origin=c]{180}{\scriptsize$#2$}}}%
  \endgroup}

\begin{document}
\title{Poker Probabilities}
\author{Alan Grayson}
\maketitle

\blackcard{10}{\clubsuit}
\blackcard{J}{\clubsuit}
\blackcard{Q}{\clubsuit}
\blackcard{K}{\clubsuit}
\blackcard{A}{\clubsuit}


\section*{Introduction}

Hi! This article is about finding the probabilities for getting each kind of poker hand.
We will assume a poker hand is five cards taken from a complete shuffled 52-card deck.
\section{Total amount of poker hands}
To figure out probability of getting a hand in poker, you need to take the amount of ways to get the hand and divide it by the total amount of hands.
In this section, We'll figure out how many total hands there are in poker.
\\\\
Since the only rule is that there will not be two cards that are
exactly the same - because two king of clubs can't be in one complete
deck - you have $52 \cdot 51 \cdot 50 \cdot 49 \cdot 48$.  You still
need to divide by 5! because order doesn't matter. So you have,


\begin{equation}\label{total hands}
{52 \cdot 51 \cdot 50 \cdot 49 \cdot 48 \over 5!} = 2596980 \hbox{ total hands}
\end{equation}
\section{Two of a kind}

In poker if you have two and only two cards that are the same number in your hand, that is a two of a kind.
For example if you have

\redcard{10}{\diamondsuit}
\redcard{10}{\heartsuit}
\blackcard{J}{\clubsuit}
\blackcard{K}{\clubsuit}
\blackcard{A}{\spadesuit}
\\\\

then you hand contains an two of an kind(two tens).
To figure out the probability you first need to know how many ways are there to get two of a kind.
We'll arrange the cards so that you have the double first.

For the first card you can have anything, so there are $52$ possibilities.

For the second card it has to be the same number, so there is one for every other suit you didn't use. $4 - 1 = 3$ possibilities.

For the third card it can be anything as long as its not the same number so you wouldn't get three of a kind. $52 - 4 = 48$ possibilities

The fourth card can be any number that didn't appear yet, so there are $44$ possibilities.

For the last card, you can have any number that didn't appear so $40$ possibilities.
You have two ways to arrange the double and six ways to arrange the other 3 cards.
So in total you have

\begin{equation}
{52 \cdot 3 \cdot 48 \cdot 44 \cdot 40 \over 2 \cdot 6} = 1098240  \hbox{ ways}
\end{equation}


Now you take the amount of ways and divide by the total amount of hands (\ref{total hands}).

\begin{equation}
{1098240 \over 2596980} = {352 \over 833} \approx  42.26\% \hbox{ chance to get an two of an kind}
\end{equation}

\section{Two Pair}

A Two pair is if you have two two of a kinds.
We'll arrange the cards so that the doubles are the first.
Now we'll find how many ways to get it.

The first card can be anything, so there are $52$ ways.

The next card has to be the same number, so there are $3$ ways.

The third card can be any new card, so there are $48$ ways.

For the fourth card, you have to have the same number as the third, so $3$ ways.

The last card can be any new number, so there are $44$ ways.

You have two ways to arrange the first pair and two ways to arrange the second one. You also can switch which pair is first so two more ways.

$ 2 \cdot 2 \cdot 2 = 8$ ways to arrange

So you have

\begin{equation}
{52 \cdot 3 \cdot 48 \cdot 3 \cdot 44 \over 8} = 123552 \hbox{ ways}
\end{equation}

divide that by the total amount of hands (\ref{total hands}) and you get

\begin{equation}
{123552 \over 2598960} = {198 \over 4165} \approx 4.75\% \hbox{ chance to get an two pair}
\end{equation}

\section{Three of a kind}

A Three of a kind is when you have three and only three of one number.
We'll arrange the cards so that the three of a kind is first.

For the first card, you can have any card, so $52$ ways for the first card.

For the next card, it has to be the same, so $3$ ways.

For the third card, it also has to be the same, so $2$ ways.

For the fouth card, it can be any new number, so $48$ ways.

For the last card, it can be any new number, so $44$ ways.

There are six ways to rearrange the 3 of a kind, and two ways to rearrange the other card.
You have

\begin{equation}
  {52 \cdot 3 \cdot 2 \cdot 48 \cdot 44 \over 6  \cdot 2} = 54912 \hbox{ total ways}
\end{equation}
Then you divide it by the total amount of ways (\ref{total hands}) and you get
\begin{equation}
  {54912 \over 2598960} = {88\over4165} \approx 2.11\% \hbox{ chance to get an three of a kind}
\end{equation}
\section{Straight}
A straight is where you have 5 consecutive cards regardless of the
suit.\footnote{Actually, if they are all the same suit, it would be a
straight flush and wouldn't be counted as a straight.}

To figure out how many ways there are to get a straight, you take the product of how many ways there are for each card.

We'll organize the cards so that it is smallest to we'll largest.

The ace can be an 1 or an 14 but not both so the first card can be from ace to $ 14 - 4 = 10$ .

The first card can also be of any suit, so $ 10 * 4 = 40 $.

The rest of the card all are one bigger so 4 ways for the rest of the cards.

However if all the cards are of the same suit, it would be a straight flush.

The first card has $ 40 $ too, but the rest just have one way.

That means you subract $  40 $ from the total amount of ways.

You have
\begin{equation}
  40 * 4 * 4 * 4 * 4 - 40 = 10200
\end{equation}

Then you divide it by the total amount of hands (\ref{total hands}).

You have

\begin{equation}
  {10200 \over 2598960} = {5 \over 1274} \approx 0.392\% \hbox{ chance to get an straight!!}
\end{equation}
0.392
\section{Flush}

If you have five cards of the same suit, Then you have an flush.

For example, if you had

\redcard{5}{\heartsuit}
\redcard{7}{\heartsuit}
\redcard{9}{\heartsuit}
\redcard{J}{\heartsuit}
\redcard{A}{\heartsuit}
\\\\

they are all hearts, so it would be a flush.

First card can be anything, so $52$ ways.

Second card is the same suit, so $12$ ways.

Third card is the same suit, so $11$ ways.

Fourth card is the same suit, so $10$ ways.

Last card is the same suit, so $9$ ways.

Order doesn't matter so 5! ways to organize.

Divide it by total amount of hands(\ref{total hands}) and you have

\begin{equation}
  {52 \cdot 12 \cdot 11 \cdot 10 \cdot 9} \over {2598960 \cdot 5!} = {33 \over 16660} \approx 0.198\%
\end{equation}
\section{Full house}
A Full house is a two of a kind and a three of a kind.
\\\\
\redcard{5}{\heartsuit}
\blackcard{5}{\clubsuit}
\blackcard{9}{\clubsuit}
\redcard{9}{\heartsuit}
\blackcard{9}{\spadesuit}
We'll arrange it so the three of an kind is first.

The first card can be anything, so $52$ ways.

The second card is the same, so $3$ ways.

The third card is also the same, so $2$ ways.

The fourth card can be any new number, so $48$ ways.

The last card is the same as the fourth card, so $3$ ways.

There are 3! ways to rearrange the three of an kind and 2! ways to rearrange the pair.

You have
\begin{equation}
  {52 \cdot 3 \cdot 2 \cdot 48 \cdot 3 \over 3! \cdot 2! \cdot 2598960} = {6 \over 4165} \approx 0.14\%
\end{equation}
\section{Four of an kind}

\blackcard{5}{\spadesuit}
\blackcard{5}{\clubsuit}
\redcard{5}{\heartsuit}
\redcard{5}{\diamondsuit}
\redcard{A}{\heartsuit}
A four of an kind is when you have four of 1 number.
\\\\
We'll arrange it so you have the 4 of a kind first.

The first card can be anything, so $52$ ways.

The second card has to be the same number, so 3 ways.

The third card also has to be the same number, so 2 ways.

The fourth card also has to be the same number, so 1 way.

The last card can be any new number, so 48 ways.

4! ways to rearrange it.

So you have
\begin{equation}
  {52 \cdot 3 \cdot 2 \cdot 1 \cdot 48 \over 2598960 \cdot 4!} ={1 \over 4165} \approx 0.024 \% \hbox{!}
\end{equation}
\section{Straight flush!!}
\blackcard{5}{\clubsuit}
\blackcard{6}{\clubsuit}
\blackcard{7}{\clubsuit}
\blackcard{8}{\clubsuit}
\blackcard{9}{\clubsuit}

In poker, the best possible kind of hand is a straight flush \footnote{royal flushes are actually just the best straight flushes}

They are when you have a straight of same suit.0.198

Organize it so the smallest card is first.

The first card can be anything from Ace to Ten, because if it started with eleven, there could only be 4 cards in the straight.
Anyways, 40 ways for the first card.

The rest of the cards only have one possibility.

No way to rearrange.

So you have:
\begin{equation}
  {40*1*1*1*1 \over 2598960} = 1 \over 649745 \approx 0.0015\% \hbox{chance}
\end{equation}
A royal flush, would have only four ways to start and has $0.00015\%$ chancce.

\section{High card}
\redcard{7}{\heartsuit}
\redcard{8}{\diamondsuit}
\blackcard{10}{\clubsuit}
\redcard{J}{\heartsuit}
\blackcard{K}{\spadesuit}
If you don't have anything in poker, that is an high card.

You just take 100\% and subtract by all the other hands.

You have
\begin{equation}
  100 - 42.26 - 4.75 - 2.11 - 0.392 - 0.198 - 0.14 - 0.024 - 0.0015 \approx 50.1245 \%
\end{equation}
\end{document}
