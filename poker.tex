\documentclass[12pt]{article}

\begin{document}
\title{Poker Probabilities}
\author{Alan Grayson}
\maketitle

\section*{Introduction}

Hi! This article is about finding the probabilities for getting each kind of poker hand.
We will assume a poker hand is five cards taken from a complete shuffled 52-card deck.
\section{Total amount of poker hands}
To figure out probability of getting a hand in poker, you need to take the amount of ways to get the hand and divide it by the total amount of hands.

Since the only rule is that there will not be two cards that are
exactly the same - because two king of clubs can't be in one complete
deck - you have $52 \cdot 51 \cdot 50 \cdot 49 \cdot 48$.  You still
need to divide by 5! because order doesn't matter. So you have,


\begin{equation}
{52 \cdot 51 \cdot 50 \cdot 49 \cdot 48 \over 5!} = 2596980 \hbox{ total hands}
\end{equation}
\section{Two of a kind}

In poker if you have two and only two cards that are the same number in your hand, that is a two of a kind.
For example if you have an 10 of diamonds, a 10 of hearts, a jack of clubs, a king of hearts, and an ace of spades then you have two of a kind.
To figure out the probability you first need to know how many ways are there to get two of a kind.
We'll arrange the cards so that you have the double first.

For the first card you can have anything, so there are $52$ possibilities.

For the second card it has to be the same number, so there is one for every other suit you didn't use. $4 - 1 = 3$ possibilities.

For the third card it can be anything as long as its not the same number so you wouldn't get three of a kind. $52 - 4 = 48$ possibilities

The fourth card can be any number that didn't appear yet, so there are $44$ possibilities.

For the last card, you can have any number that didn't appear so $40$ possibilities.
You have two ways to arrange the double and six ways to arrange the other 3 cards.
So in total you have

\begin{equation}
{52 \cdot 3 \cdot 48 \cdot 44 \cdot 40 \over 2 \cdot 6} = 1098240  \hbox{ ways}
\end{equation}


Now you take the amount of ways and divide by the total amount of hands (1).

\begin{equation}
{1098240 \over 2596980} = {352 \over 833} = \hbox{ about } 42.26\% \hbox{ chance to get an two of an kind}
\end{equation}

\section{Two Pair}

A Two pair is if you have two two of a kinds.
We'll arrange the cards so that the doubles are the first.
Now we'll find how many ways to get it.

The first card can be anything, so there are $52$ ways.

The next card has to be the same number, so there are $3$ ways.

The third card can be any new card, so there are $48$ ways.

For the fourth card, you have to have the same number as the third, so $3$ ways.

The last card can be any new number, so there are $44$ ways.

You have two ways to arrange the first pair and two ways to arrange the second one. You also can switch which pair is first so two more ways.

$ 2 \cdot 2 \cdot 2 = 8$ ways to arrange

So you have

\begin{equation}
{52 \cdot 3 \cdot 48 \cdot 3 \cdot 44 \over 8} = 123552 \hbox{ ways}
\end{equation}

divide that by the total amount of hands (1) and you get

\begin{equation}
{123552 \over 2598960} = {198 \over 4165} = \hbox{ about } 4.75\% \hbox{ chance to get an two pair}
\end{equation}

\section{Three of a kind}

A Three of a kind is when you have three and only three of one number.
We'll arrange the cards so that the three of a kind is first.

For the first card, you can have any card, so $52$ ways for the first card.

For the next card, it has to be the same, so $3$ ways.

For the third card, it also has to be the same, so $2$ ways.

For the fouth card, it can be any new number, so $48$ ways.

For the last card, it can be any new number, so $44$ ways.

There are six ways to rearrange the 3 of a kind, and two ways to rearrange the other card.
You have

\begin{equation}
  {52 \cdot 3 \cdot 2 \cdot 48 \cdot 44 \over 6  \cdot 2} = 54912 \hbox{ total ways}
\end{equation}
Then you divide it by the total amount of ways (1) and you get
\begin{equation}
  {54912 \over 2598960} = {88\over4165} = \hbox{ about } 2.11\% \hbox{ chance to get an three of a kind}
\end{equation}
\section{Straight}
\end{document}
